\documentclass{resume}

%%%%%%%%%%%%%%%%%%%%%%%%%%%%%%%%%%%%%%%%%

\begin{document}
\begin{center}
	\MakeUppercase{\LARGE \bf Francisco Fernando Valdivia Catata}
	\vspace{2mm} \\
	\MakeUppercase{Bachiller en Economía $|$ Tercio Superior}
	\vspace{2mm} \\
	{\color{vino}\faPhoneSquare} {\href{tel:934395428}{(+51) 934-395-428}} $|$
	{\color{vino}\faEnvelopeSquare} {\href{mailto:francisco.valdivia.catata@gmail.com}{francisco.valdivia.catata@gmail.com}} $|$
	{\color{vino}\faLinkedinSquare} {\href{https://www.linkedin.com/in/francisco-valdivia-catata/}{@francisco-valdivia-catata}}
	\vspace{2mm}
\end{center}

%%%%%%%%%%%%%%%%%%%%%%%%%%%%%%%%%%%%%%%%%

{\color{vino} \noindent\MakeUppercase{\large \bf Resumen Ejecutivo} \\
\rule[3mm]{\textwidth}{0.5mm}}

\noindent Bachiller en Economía con mención en Economía Pública por la Universidad Nacional Mayor de San Marcos.
Con especialización en Regulación Económica y Políticas de Competencia.
Profesional con sólidos conocimientos en Teoría Económica y Econometría Aplicada.
Dominio del idioma inglés y manejo de softwares econométricos, anális de datos e investigación.
Interés en el manejo de bases de datos, formulación de indicadores socioeconómicos y desarrollo de modelos econométricos
en los campos del mercado laboral, educación, pobreza y desigualdad.

\vspace{5mm}

%%%%%%%%%%%%%%%%%%%%%%%%%%%%%%%%%%%%%%%%%

{\color{vino} \noindent\MakeUppercase{\large \bf Experiencia Extraprofesional} \\
\rule[3mm]{\textwidth}{0.5mm}}

\noindent {\color{vino} \faBriefcase} \; {\bf Director del Área Académica} \hfill
{\color{vino}\faCalendarCheckO} {Agosto 2023 - Actualidad} \\
{\color{vino}\faInstitution} \, {Evidencia: Observatorio de Políticas Públicas para el Desarrollo}

\vspace{2mm}

\noindent Asociación sin fines de lucro dedicada a la investigación socioeconómica del impacto de las políticas públicas en el Perú.
Actualmente a cargo de actividades de gestión, coordinación y revisión de reportes estadísticos,
artículos académicos, productos editoriales y talleres en colaboración con diversos profesionales e investigadores.

\vspace{2mm}

\noindent {\color{vino} \faBriefcase} \; {\bf Asistente de Cátedra en Regulación} \hfill
{\color{vino}\faCalendarCheckO} {Abril 2024 - Actualidad} \\
{\color{vino}\faInstitution} \, {Universidad Nacional Mayor de San Marcos}

\vspace{2mm}

\noindent Actualmente a cargo de diseñar y desarrollar sesiones introductorias en teoría de la regulación,
regulación en infraestructura de transporte de uso público y sector energético.

\vspace{5mm}

%%%%%%%%%%%%%%%%%%%%%%%%%%%%%%%%%%%%%%%%%

{\color{vino} \noindent\MakeUppercase{\large \bf Extensión Universitaria} \\
\rule[3mm]{\textwidth}{0.5mm}}

\noindent {\color{vino} \faMortarBoard} \, {\bf Programa de Extensión Universitaria 2024 - INDECOPI} \hfill
{\color{vino}\faCalendarCheckO} {2024} \\
{\color{vino} \faInstitution} \; {Instituto Nacional de Defensa de la Competencia y Protección de la Propiedad Intelectual}

\vspace{3mm}

\noindent {\color{vino} \faMortarBoard} \, {\bf XX Curso de Extensión Universitaria - OSITRÁN} \hfill
{\color{vino}\faCalendarCheckO} {2023} \\
{\color{vino}\faInstitution} \; {Organismo Supervisor de la Inversión en Infraestructura de Transporte de Uso Público}

\vspace{3mm}

\noindent {\color{vino}\faMortarBoard} \, {\bf Advanced Economics Program 2023 - RIEF} \hfill
{\color{vino}\faCalendarCheckO} {2023} \\
{\color{vino}\faInstitution} \; {Research Institute in Economics and Finance}

\vspace{5mm}

%%%%%%%%%%%%%%%%%%%%%%%%%%%%%%%%%%%%%%%%%

{\color{vino} \noindent\MakeUppercase{\large \bf Capacitaciones Técnicas} \\
\rule[3mm]{\textwidth}{0.5mm}}

\vspace{-3mm}

\begin{table}[H]
	\begin{tabular}{lll}
		{\bf Redacción de artículos científicos} $|$ {\it Universidad Peruana Cayetano Heredia} & \hspace{7.5mm} & {\color{vino}\faCalendarCheckO} {Agosto, 2023} \\
		{\bf Econometría aplicada y procesamiento de datos} $|$ {\it Valora Consultores} &  & {\color{vino}\faCalendarCheckO} {Abril, 2023} \\
		%{\bf Modelamiento econométrico aplicado} $|$ {\it Grupo Lambda} &  & {\color{vino}\faCalendarCheckO} {Mayo, 2021} \\
		{\bf Análisis y manejo de bases de datos} $|$ {\it Neuroscience} &  & {\color{vino}\faCalendarCheckO} {Marzo, 2020}
	\end{tabular}
\end{table}

%\noindent {\bf Redacción de artículos científicos} $|$ {\it Universidad Peruana Cayetano Heredia} \hfill  {Agosto, 2023} \\
%{\bf Microeconometría aplicada y procesamiento de datos} $|$ {\it Valora Consultores} \hfill {Abril, 2023} \\
%{\bf Modelamiento econométrico aplicado} $|$ {\it Grupo Lambda} \hfill {Mayo, 2021} \\
%{\bf Análisis y manejo de bases de datos} $|$ {\it Neuroscience 1.0} \hfill {Marzo, 2020}

%\vspace{5mm}

%%%%%%%%%%%%%%%%%%%%%%%%%%%%%%%%%%%%%%%%%

{\color{vino} \noindent\MakeUppercase{\large \bf Habilidades profesionales} \\
\rule[3mm]{\textwidth}{0.5mm}}

\vspace{-3mm}

\begin{table}[H]
	\begin{tabular}{ll}
		{\bf Idiomas}: & Inglés ({\it lectura, escritura y expresión oral a nivel intermedio}) \\
		{\bf Ofimática}: & Google Workspace, Microsoft Office ({\it hojas de cálculo y presentaciones a nivel avanzado}) \\
		{\bf Programas}: & \LaTeX, Stata, R, Python ({\it redacción de informes y manejo de datos a nivel intermedio}) \\
		{\bf Capacidades}: & Orientación al usuario, Gestión de información, Trabajo en equipo, Capacidad de análisis \\
		{\bf Pasatiempos}: & Gestión de conocimiento, Pedagogía, Programación orientada a objetos
	\end{tabular}
\end{table}

{\color{vino} \noindent\MakeUppercase{\large \bf Referencias} \\
\rule[3mm]{\textwidth}{0.5mm}}

\vspace{-3mm}

\begin{table}[H]
	\begin{tabular}{ll}
		{\color{vino}\faUser} \, {\bf Ronald Nilton Silva Gil} & {\color{vino}\faUser} \, {\bf Sandro Alejandro Huamaní Antonio} \\
		{Asesor de la Gerencia General - ONP} & {Director de Regulación Tarifaria - SUNASS} \\
		{\it Dr. en Gobierno y Política Pública} & {\it Mg. in Public Policy} \\
		{rsilvagil@gmail.com} & {shuamani@sunass.gob.pe}
	\end{tabular}
\end{table}

\begin{multicols}{2}
	
	
	
\end{multicols}

\end{document}